\documentclass[11pt]{article}

\usepackage[margin=0.75in,paperwidth=8.5in,paperheight=11in]{geometry}

%packages for math papers
\usepackage{amsmath,amsfonts,amssymb}
%to prevent latex form using hyphenated words
\usepackage[none]{hyphenat}
%for title page to have a custom header and footer
\usepackage{fancyhdr}

\pagestyle{fancy}
%to clear the default header and footer
\fancyhead{}
\fancyfoot{}
%To have make new header
%[L] means left aligned
\fancyhead[L]{\textit{\MakeUppercase{My Title will come here}}}
\fancyhead[R]{\textit{\MakeUppercase{Shashank}}}
%to add the page number of the page
\fancyfoot[C]{\thepage}
%to remove the horizontal line
%\renewcommand{\headrulewidth}{0pt}
%you can also create or remove line in header and footer
%\renewcommand{\footrulewidth}{0pt}

\usepackage{graphicx}
\usepackage{float}
%to remove indentation
%\parindent 0ex
%to change the width of the indent
%\setlength{\parindent}{4em}
%to change the amount of space between paragraphs
%\setlength{\parskip}{1em}
%to change the line spacing(space between lines) of paragraphs
\renewcommand{\baselinestretch}{1.5}

%to add table of contents
\usepackage[nottoc,notlot,notlof]{tocbibind}

\begin{document}


The function $f(x) = (x-3)^2+ \frac{1}{2}$ has domain $ \mathrm{D}_f:(-\infty,\infty)$
%Here \mathrm is used to unitalicize D
and range $\mathrm{R}_f:\left[\frac{1}{2},\infty\right)$

\vspace{1in}
limits:

%\to is for arrow
$\lim_{x\to a}$

$\lim\limits_{x\to a^+}f(x)$

$\lim\limits_{x \to a} \frac{f(x)-f(a)}{x-1} = f'(a)$

%to have proper sizing
$\displaystyle{\lim\limits_{x \to a} \frac{f(x)-f(a)}{x-1} = f'(a)}$

$\int \sin x dx$

$\int \sin x \,dx = -\cos x +c $ 
%\, forces a space

%to control the size of integral
$\displaystyle{\int \sin x \,dx = -\cos x +c }$

one way: $\int_a^b$

another way: $\int\limits_a^b$

$\displaystyle{\int_a^b}$

$\displaystyle{\int\limits_a^b}$

$\displaystyle{\int\limits_{2a}^{3b}x^2 \,dx = \left[\frac{x^3}{3}\right]_{2a}^{3b}}$

\vspace{1in}
Summation notation

$\sum$

$\displaystyle{\sum\limits_{n=1}^{\infty} ar^n = a + ar + \cdots + ar^n}$
%\cdot mean single dot and cdots means three dots
$\displaystyle{\int_a^b f(x) \,dx = \lim\limits_{x \to \infty} \sum\limits_{k=1}^{n} f(x,k) \cdot \Delta x}$
%\delta means small delta and Delta means capital delta

\vspace{1in}
vector notation

$\vec{v}=v_1 \vec{i}+v_2\vec{i} = \langle v_1,v_2 \rangle$

\vspace{1in}
creating a math paper

It should include:
\begin{itemize}
\item title page
\item table of contents
\item section and subsections
\item footnotes
\item citations
\item biblography
\end{itemize}

\begin{titlepage}
\begin{center}
\vspace{1in}
\Large\textbf{Latex practice}\\
\Large\textbf{Just and exercise}\\
%vfil is IMPORTANT
%automatically adjusts the space between the two items and this space keeps decreasing as the the lower text keeps increasing
\vfil
\line(1,0){400}\\[1mm]
\huge{\textbf{Use of vfil is important}}\\[3mm]
\Large{\textbf{This is sampel sub title}}\\[1mm]
\line(1,0){400}\\
%as the text is very low we can add more space below the title using vfil
\vfil
By Shashank\\
Candidate \\
\today

\end{center}
\end{titlepage}

%compline twice
\tableofcontents
%thispagestyle SHOULD COME BEFORE clear
%to make the table of contents to line on a separate page
\thispagestyle{empty}
%to remove the header and footer on this page
\clearpage

%to change the page numbers as we want
\setcounter{page}{1}



\section{Introduction}
\section{Scoring Criteria}
\subsection{Communication}
\subsection{Mathematical Presentation}
\subsection{Personal Engagement}
\subsection{Reflection}
\subsection{Use of Mathematics}
\section{Conclusion}
\section{\LaTeX}

\begin{table}[H]
\centering
\begin{tabular}{|c|c|}\hline
x & 1 \\ \hline
2 & 3\\ \hline

\end{tabular}
%label for reference
\label{tab:data1}
\caption{Caption goes here}
\end{table}

\begin{figure}[H]
\centering
\includegraphics[scale=0.25]{iron_throne}
\label{fig:data2}
\caption{Caption goes here}


\end{figure}
%this will add a footnote
I'll be adding a footnote here. \footnote{An exmaple footnote}

%This will add a refernce to a label figure
I'll abe adding a reference here. See table \ref{tab:data1}

%This will add a refernce to a label to a figure
I'll abe adding a reference here. See figure \ref{fig:data2}

%This will add a citation
I'll abe adding a citation here \cite{first}.

\pagebreak



Now we'll work on bibliography


\begin{thebibliography}	
\bibitem{first}
Shashank, singh
``high school.''
\textit{practice}
web 27 2015

\end{thebibliography}









\end{document}