\documentclass[11pt]{article}
%This space before the document starts is called the Preamble
\usepackage{amsfonts,amssymb,amsmath}
%to implement the position in table
\usepackage{float}
%to implement changes in enumerate
\usepackage{enumerate}
%To change the indentation
\parindent 0px
\pagestyle{empty}


\begin{document}

This distributive property states that.
$a(b+c) = ab + ac$, for all $a,b,c \in \mathbb{R}$\\[6pt]
The equivalence class of $a$ is $[a]$

The set $A$ is defined to be ${1,2,3}$

The set $A$ is defined to be $\{1,2,3\}$

The movies tickets costs \$11.50.

The movies tickets costs $\$11.50$.

$$2(\frac{1}{x^2-1})$$

$2(\frac{1}{x^2-1})$

$$2\left[\frac{1}{x^2-1}\right)$$
 
%We still need the backslash before curly brackets
$$2\left\{\frac{1}{x^2-1}\right\}$$

$$2\left\langle\frac{1}{x^2-1}\right\rangle$$

$$2\left|\frac{1}{x^2-1}\right|$$

$$\frac{dy}{dx}|_{x=1}$$
%This wouldn't work $$\frac{dy}{dx}\right|_{x=1}$$ without \left
%By using a dot we can tell it to not display a symbol
$$\left.\frac{dy}{dx}\right|_{x=1}$$

$$\left(\frac{1}{1+\left(\frac{1}{x+1}\right)}\right)$$

Tables:\\
%here c doesn't stand for column it stands for center align, similarily l-left align r - right |symbol is 
\begin{tabular}{cccccc}
x &1 & 2 & 3 & 4 & 5\\
1 &1 & 2 & 3 & 4 & 5\\
\end{tabular}

\vspace{1cm}

\begin{tabular}{|c||c|c|c|c|c|}
\hline
x &1 & 2 & 3 & 4 & 5\\ \hline
$f(x)$ & $\frac{1}{2}$ & 11 & 12 & 13 & 14\\\hline
\end{tabular}

%If we dont type [H] the compiler will decide the best position for the table
\begin{table}[H]
%centering the table
\centering
%To increase the size of each row
\def\arraystretch{2}

\begin{tabular}{|c||c|c|c|c|c|}
\hline
x &1 & 2 & 3 & 4 & 5\\ \hline
$f(x)$ & $\frac{1}{2}$ & 11 & 12 & 13 & 14\\\hline
\end{tabular}
%caption can be also be added before tabular and it is centered so ensure that the table is aldo centererd
\caption{These value represent the function $f(x)$}
\end{table}

\begin{table}[H]
\centering
\caption{These value represent the function $f(x)$}
\def\arraystretch{2}

\begin{tabular}{|c|c|}
\hline
$f(x)$ & $fx(x)$\\ \hline
$x>0$ & The function $f(x)$ is increasing. The function $f(x)$ is increasing. The function $f(x)$ is increasing. The function $f(x)$ is increasing. The function $f(x)$ is increasing.\\\hline
\end{tabular}

\end{table}

\begin{table}[H]
\centering
\caption{These value represent the function $f(x)$}
\def\arraystretch{2}

%The parameter for the 'p'(paragraph is the width of the paragarph)
\begin{tabular}{|l|p{4cm}|}
\hline
$f(x)$ & $fx(x)$\\ \hline
$x>0$ & The function $f(x)$ is increasing. The function $f(x)$ is increasing. The function $f(x)$ is increasing. The function $f(x)$ is increasing. The function $f(x)$ is increasing.\\\hline
\end{tabular}
\end{table}

Equation array:
%one way-
%\begin{eqnarray}
%\end{eqnarray}

%preferred way-
\begin{align}
%everything in align is in math mode
%so if you actually want text use \text{}
5x^2 \text{ place your words here} \\
5x^2-9 = x+3 \\
5x^2-x-12 = 0
%To align with the equal to sign
5x^2-9 &= x+3 \\
5x^2-x-12 &= 0 \\
&=12 +5-5x^2
\end{align}

\begin{align*}
5x^2-9 &= x+3 \\
5x^2-x-12 &= 0 \\
&=12 +5-5x^2
\end{align*}

\begin{align}
5x^2-9 &= x+3 \\
5x^2-x-12 &= 0 \\
&=12 +5-5x^2
\end{align}

lists

\begin{enumerate}
\item pencil
\item calculater
\item ruler
\item notebook
\end{enumerate}

\begin{itemize}
\item pencil
\item calculater
\item ruler
\item notebook
\end{itemize}

\begin{enumerate}
\item pencil
\item calculater
\item ruler
\item notebook
	\begin{enumerate}
	%It can work without indentation also
	\item notes
	\item homework
	\item assignments
		\begin{enumerate}
		\item tests
		\item quizzes
		\item journal entries
	
		\end{enumerate}
	\end{enumerate}
\item highlighters
\end{enumerate}

\vspace{1cm}

%[A.] is used to change how we enumerate
\begin{enumerate}[A.]
\item pencil
\item calculater
\item ruler
\item notebook
	\begin{enumerate}
	%It can work without indentation also
	\item notes
	\item homework
	\item assignments
		\begin{enumerate}
		\item tests
		\item quizzes
		\item journal entries
	
		\end{enumerate}
	\end{enumerate}
\item highlighters
\end{enumerate}

\begin{enumerate}[i.]
\item pencil
\item calculater
\item ruler
\item notebook
\end{enumerate}

%to begin with the number 6 we give the last value as the parameter, in this case, 5
\begin{enumerate}\setcounter{enumi}{5}
\item pencil
\item calculater
\item ruler
\item notebook
\end{enumerate}

%used to add a new page
\pagebreak

\begin{itemize}
\item pencil
\item calculater
\item ruler
\item notebook
	\begin{itemize}
	%It can work without indentation also
	\item notes
	\item homework
	\item assignments
		\begin{itemize}
		\item tests
		\item quizzes
		\item journal entries
	
		\end{itemize}
	\end{itemize}
\item highlighters
\end{itemize}

%you can also add a inch
\vspace{1in}

\begin{enumerate}
\item[you can add anything here] pencil
\item calculater
\item[] ruler
\item[] notebook
\item highlighters
\end{enumerate}



\end{document}