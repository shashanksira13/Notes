\documentclass[11pt]{article}
%to add clickable hyperlink reference 
\usepackage{hyperref}
%to decrease the size of the margins. It is automatically applied 
\usepackage{fullpage}
%the fullpage package still leaves a lot of space at the top and the bottom, we can use geometry package, it gives more control. have 1 inch margin is a standard
%\usepackage[top=1in,bottom=1in,left=1in,right=1in]{geometry}
%another way- In this we add a reference point
%\usepackage[margin=1in,paperwidth=8.5in,paperheight=11in]{geometry}


%Add some text formatting
\usepackage{amsfonts}


%This is called macros, it is used when we have to use one particular thing several times
\def\eq1{y=\frac{x}{3x^2+x+1}}

\def\labelaxes{Remember to include a scale and label your axes}

%to use graphics like adding an image
\usepackage{graphicx}

%Add titles
%The backslash after the \LaTeX'\' is used to add space
\title{My \LaTeX\ Document}
\author{Shashank}
\date{\today}
%You can add anything in the date section not just necessarily a date

\begin{document}
%to add table of contents..
%it requires to build two times
\tableofcontents

%to add the title author and date
\maketitle


This will produce \textit{italicized} text.
This will produde \textbf{bold face} text.
This will produce \textsc{small caps} text.
This will produce \texttt{typewriter font} text.
%typewriter font is monospaced so can be used for url and stuff.
Please visit Shashank's website at \url{http://michellekrummel.com}.

Another way to add the clickable hyperlink reference is \href{http://michellekrummel.com}{My website}.

\vspace{1cm}
Changing the text size

Please excuse my \begin{tiny}aunt Sally. \end{tiny}

Please excuse my \begin{scriptsize}aunt Sally. \end{scriptsize}

Please excuse my \begin{small}aunt Sally. \end{small}

Please excuse my \begin{normalsize}dear aunt Sally. \end{normalsize} 

Please excuse my dear aunt Sally.

Please excuse my \begin{large}dear aunt Sally.
\end{large} 

Please excuse my \begin{Large}dear aunt Sally.
\end{Large} 

Please excuse my \begin{huge}dear aunt Sally.
\end{huge} 

Please excuse my \begin{Huge}dear aunt Sally.
\end{Huge} 


\vspace{1cm}
\begin{center}This line is centered\end{center}

\begin{flushleft}
This line is left-justified
\end{flushleft}

\begin{flushright}
This line is right-justified
\end{flushright}

%If we use \centering then everything after that is affected same goes for \Large 
\Large
This line is centered.\\
This line is centered again
%Use \section*{sdfd} to not number it
\section*{Linear Functions}
	\subsection*{Slope-Intercept Form}
		\subsubsection*{Example1:}
		\subsubsection{Exmaple2:}
	\subsection{Standard Form}
	\subsection{Point-Slope Form}
\section*{Quadratic Functions}
	\subsection{Vertex Form}
	\subsection{Standard Form}
\section{Quadratic Functions}
	\subsection{Vertex Form}
	\subsection{Standard Form}

\vspace{1in}

Use of packages

The set of the Natural numbers is denoted by
$\mathbb{N}$.

The set of Integers is denoted by $\mathbb{Z}$

The set of Real numbers is denoted by $\mathbb{R}$

\vspace{1in}

Use of macros

Graph $\eq1$

Identify the aymptotes for the graph of $eq1$

\labelaxes

\vspace{1in}
Working with garphics

image has to be saved in the same image as the latex file
we can only use .png .jpg , .gif or .pdf files
names should not have spaces
jpeg files are written as jpg


%image has to be saved in the same image as the latex file
% we can only use .png .jpg , .gif or .pdf files
% names should not have spaces
% jpeg files are written as jpg
\includegraphics[scale=0.1]{iron_throne.jpg}
%scaling is relative
\includegraphics[width=2in]{iron_throne.jpg}\\
\begin{center}
\includegraphics[width=2in]{iron_throne.jpg}
\end{center}
\begin{center}
\includegraphics[width=2in,angle=45]{iron_throne.jpg}
\end{center}

\vspace{1in}
Advices for debuggin and error

use line numbers

go line by line

use $\left( with \right)$

make sure of all brackets and \$ and enumerate are closed

Search messages of the error on google if you cant solve the error

\vspace{1in}

tips for texmaker

you can use an external pdf viewer from configure texmaker

you add various types of symbols from the tray that opens on the left when you click on the structures tab from lower left corner 

you can also choose from comands from the left array in the default mode, when the structure is off and also from the latex and the math menu

while making a new file we can choose quick start from wizard to make your life easy

you can create a table and array form quick tabular or array from wizard

we can also create user tags from user just like macros but we cannot create already used tags like title

you can use latex reference and user from help


\end{document}
